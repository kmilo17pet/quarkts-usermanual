\subsection{Input groups for edge-checking}

An interface to manage and simplify the value(with edge-checking) of incoming digital signals groups.
\medskip

\begin{lstlisting}[style=CStyle]
qBool_t qEdgeCheck_Setup( qEdgeCheck_t * const Instance, 
                          const qCoreRegSize_t RegisterSize,
                          const qClock_t DebounceTime )
\end{lstlisting}

Initialize an I/O edge-check instance. \index{\lstinline{qEdgeCheck_Setup}}

\subsubsection*{Parameters:}
\begin{itemize}
    \item \lstinline{Instance} : A pointer to the I/O edge-check object.
    \item \lstinline{RegisterSize} : The specific-core register size: \lstinline{QREG_8BIT}, \lstinline{QREG_16BIT} or \lstinline{QREG_32BIT}(default).
    \item \lstinline{DebounceTime} : The specified time (in epochs) to bypass the bounce of the input nodes.
\end{itemize}

\subsubsection*{Return value:}
\lstinline{qTrue} on success, otherwise returns \lstinline{qFalse}.

\noindent\hrulefill

\begin{lstlisting}[style=CStyle]
qBool_t qEdgeCheck_Add_Node( qEdgeCheck_t * const Instance, 
                             qEdgeCheck_IONode_t * const Node, 
                             void *PortAddress, 
                             const qBool_t PinNumber )
\end{lstlisting}

Inserts an I/O node to the edge-check instance. \index{\lstinline{qEdgeCheck_Add_Node}}

\subsubsection*{Parameters:}
\begin{itemize}
    \item \lstinline{Instance} : A pointer to the I/O edge-check object.
    \item \lstinline{Node} :  A pointer to the input-node object.
    \item \lstinline{PortAddress} : The address of the core PORTx-register to read the levels of the specified \lstinline{PinNumber}.
    \item \lstinline{PinNumber} : The specified pin to read from \lstinline{PortAddress}.
\end{itemize}

\subsubsection*{Return value:}
\lstinline{qTrue} on success, otherwise returns \lstinline{qFalse}.


\noindent\hrulefill

\begin{lstlisting}[style=CStyle]
qBool_t qEdgeCheck_Update( qEdgeCheck_t * const Instance )
\end{lstlisting}

Update the status of all nodes inside the I/O edge-check instance (non-blocking call). \index{\lstinline{qEdgeCheck_Update}}

\subsubsection*{Parameters:}
\begin{itemize}
    \item \lstinline{Instance} : A pointer to the I/O edge-check object.
\end{itemize}

\subsubsection*{Return value:}
\lstinline{qTrue} on success, otherwise returns \lstinline{qFalse}.


\noindent\hrulefill

\begin{lstlisting}[style=CStyle]
qBool_t qEdgeCheck_Get_NodeStatus( const qEdgeCheck_IONode_t * const Node )
\end{lstlisting}

Query the status of the specified input-node. \index{\lstinline{qEdgeCheck_Get_NodeStatus}}

\subsubsection*{Parameters:}
\begin{itemize}
    \item \lstinline{Instance} : A pointer to the I/O edge-check object.
\end{itemize}

\subsubsection*{Return value:}
The status of the input node : \lstinline{qTrue}, \lstinline{qFalse}, \lstinline{qRising}, \lstinline{qFalling} or \lstinline{qUnknown}.