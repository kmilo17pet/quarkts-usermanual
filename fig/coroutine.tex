\begin{figure}[H]
    \centering
    \begin{tikzpicture}[x=0.75pt,y=0.75pt,yscale=-1,xscale=1,scale=1]
        \foreach \x in {100,120,160,180,220}{ \draw    (121,\x) -- (191,\x) ; }
        \foreach \x in {100,120,140,180,200,220}{ \draw    (331,\x) -- (401,\x) ; }
        \draw   (290,70) -- (430,70) -- (430,250) -- (290,250) -- cycle ;
        \draw   (80,70) -- (220,70) -- (220,250) -- (80,250) -- cycle ;
        
        \draw  [fill=lightgray  ,fill opacity=1 ] (290,46) .. controls (290,42.69) and (292.69,40) .. (296,40) -- (424,40) .. controls (427.31,40) and (430,42.69) .. (430,46) -- (430,70) .. controls (430,70) and (430,70) .. (430,70) -- (290,70) .. controls (290,70) and (290,70) .. (290,70) -- cycle ;
        \draw  [fill=lightgray  ,fill opacity=1 ] (80,46) .. controls (80,42.69) and (82.69,40) .. (86,40) -- (214,40) .. controls (217.31,40) and (220,42.69) .. (220,46) -- (220,70) .. controls (220,70) and (220,70) .. (220,70) -- (80,70) .. controls (80,70) and (80,70) .. (80,70) -- cycle ;
        \draw    (210,90) -- (210,130) -- (310,90) -- (310,150) -- (210,140) -- (210,200) -- (310,160) -- (310,230) -- (210,210) -- (210,238) ;
        \draw [shift={(210,240)}, rotate = 270] [fill=black  ][line width=0.75]  [draw opacity=0] (8.93,-4.29) -- (0,0) -- (8.93,4.29) -- cycle    ;
        \draw [shift={(210,90)}, rotate = 90] [color=black  ][fill=black  ][line width=0.75]      (0, 0) circle [x radius= 3.35, y radius= 3.35]   ;
        \draw (145,55) node  [align=left] {Task A};
        \draw (127,86) node [scale=0.7, font=\ttfamily] [align=left] {\ttfamily{qCR_Begin\{}};
        \draw (127,234) node [scale=0.7] [align=left] {\ttfamily{\}qCR_End;}};
        \draw (141.5,136) node [scale=0.7] [align=left] {\ttfamily{qCR_Yield;}};
        \draw (141.5,204) node [scale=0.7] [align=left] {\ttfamily{qCR_Yield;}};
        \draw (355,55) node  [align=left] {Task B};
        \draw (337,86) node [scale=0.7] [align=left] {\ttfamily{qCR_Begin\{}};
        \draw (337,234) node [scale=0.7] [align=left] {\ttfamily{\}qCR_End;}};
        \draw (350.5,156) node [scale=0.7] [align=left] {\ttfamily{qCR_Yield;}};
    \end{tikzpicture}
    \caption{Coroutines in QuarkTS}
    \label{fig:coroutine}
\end{figure}